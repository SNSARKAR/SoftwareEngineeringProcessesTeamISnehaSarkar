\documentclass[12pt]{report}
\usepackage[utf8]{inputenc}


\title{Problem 2}
\author{Sneha Sarkar}
\date{July 2019}

\begin{document}
\maketitle
\section{Requirements}The requirements for the function, F9 : $x^y$ are given below : \\
\subsection{R01 :} When the function is being calculated, if the input values for the variable 'x' and 'y' are non numerical values such as string, the system shall give an error message as the output.
\subsection{R02 :} When the function is being calculated,if the input values for the variables 'x' and 'y' are real values ,the system shall provide a real number based on the inputs and calculation.
\subsection{R03 :} When the function is being calculated,if the input value for the variable 'x' is  less than zero and the input for variable y is a odd number ,the system shall give an output which is a real number and less than zero.
\subsection{R04 :} When the function is being calculated,if the input value for the variable 'x' is  greater than zero and the input for variable y is a negative number ,the system shall give an output which is a real number and less than zero.
\subsection{R05 :} When the function is being calculated,if the input value for the variable 'x' is zero and the input for variable y is a negative number or equal to zero,the system shall give a message that the output is undefined  .
\subsection{R06 :} When the function is being calculated,if the input value for the variable 'x' is less zero and the input for variable y is a real number(not an integer value),the system shall give a message that the output is undefined  .
.
\section{Assumptions}The assumptions for the function, F9 : $x^y$ are given below :
\subsection{A01 :}The inputs for the variables of x and y can only be positive or negative real numbers, they cannot include complex numbers or trigonometric functions as they can contain infinite number of roots as outputs .
\subsection{A02 :}The input for the variable of x cannot be less than zero if the input for variable y is real number(fractional number) as the output will be a complex number as roots.
\subsection{A03 :}If the input for variable x is zero and the input for variable y is equal to zero or less than zero, the answer to the function can be 1(when x=0,y=0) or leaving the expression undefined,based on the context.
\subsection{A04 :}The inputs and output for the variables of x and y shall be in the  number format with an implied decimal point.
\begin{thebibliography}{}
\bibitem{}http://www.biology.arizona.edu/biomath/tutorials/Power/Powerbasics.html
\bibitem{}https://mathinsight.org/exponential\_function


\end{thebibliography}
\end{document}
