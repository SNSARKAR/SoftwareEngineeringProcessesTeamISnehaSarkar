\documentclass[10pt]{article}
\usepackage[utf8]{inputenc}
\begin{document}

\large\textbf{Sneha Sarkar}\\ 
\indent\Large \textbf{40070778}\\
\indent\Large \textbf{26th July,2019}\\
\indent\Large\textbf{Problem 4}\\
\section{Debugging Tool :}
The Debugging tool used for the source code of the Function F9 : $x^y$ is the in-built debugger provided by the Eclipse IDE. The Eclipse Debugger tool allows the user to run the program code interactively while observing the source code and the state of the variables during execution.It helps the user to determine the statement or points in the source code which results into the wrong output from the execution of the program code.It consists of features such as Breakpoints.Watch Points,Exceptional Break points and perspectives such as Debug view,console view,break point view,etc which aid in debugging.\\
The advantages of Eclipse IDE Debugger Tool are:
\begin{itemize}
  \item It is not necessary to rebuild/restart your debug session to obtain additional information.
  \item It organizes imports and makes navigation,error debugging(to easily navigate to error line) easier.
\end{itemize}
The disadvantages of Eclipse IDE Debugger Tool are:
\begin{itemize}
  \item The Eclipse debugger tool is most reliable when used with small scale projects.
  \item Eclipse IDE Tool is not efficient when the program code or project involves multi thread processes.
\end{itemize}
\section{Code Analysis Tool :}
The Code Analysis tool used for Function F9:$x^y$ source code is CheckStyle tool.It is a development tool which helps the developers to write a java code that adheres to the coding standards.It is a static analysis tool that automates the process of checking the Java code.It checks various aspects of the code.It can determine class design problems,method design problems.It also can also check code layout and formatting issues. 
The advantages of CheckStyle Tool are:
\begin{itemize}
  \item It is highly configurable and can be made to support almost any coding standard(such as Sun code,Google Java Style).CheckStyle has a wide variety of rules and has the ability to create custom rules also.
  \item It is portable between IDEs and is easier to integrate with external tools  since it was designed as a standalone framework.
\end{itemize}
The disadvantage of CheckStyle Tool are:
\begin{itemize}
  \item CheckStyle tool is a single file static analysis tool,which is mainly limited to the presentation of the code and does not confirm the correctness/completeness of the code .
\end{itemize}
\begin{thebibliography}{}
\bibitem{}https://www.vogella.com/tutorials/EclipseDebugging/article.html
\bibitem{}https://www.eclipse.org/community/eclipse\_newsletter/2017/june/article1.php
\bibitem{}https://softwareengineering.stackexchange.com/questions/168540/what-are-the-advantages-of-using-the-java-debugger-over-println
\bibitem{}https://checkstyle.sourceforge.io/writingchecks.html\#Limitations
\bibitem{}https://stackoverflow.com/questions/13644624/advantage-of-using-checkstyle-rather-than-using-eclipse-built-in-code-formatter


\end{thebibliography}
 
 
\end{document}